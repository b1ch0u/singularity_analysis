\documentclass[../main.tex]{subfiles}

\begin{document}

\section{Transfer theorems}

The content of this section is from \cite{Flajolet2009}, section VI.

\begin{definition}{{$\Delta(\phi, R)$}}
	For any positive numbers $\phi$ and $R$ such that $R > 1$ and $\phi \in \left] 0, \frac{\pi}{2} \right[$, we define
	\begin{equation*}\label{def_Delta}
	\Delta(\phi, R) := \{z \mid |z| < R, z \neq 1 \text{ and } |\arg(z - 1) > \phi \}
	\end{equation*}
	
	A domain $D$ such that there exists $\phi$ and $R$ as above and $D = \Delta(\phi, R)$ will be called a \emph{$\Delta$-domain}.
	
	A function that is analytic on some $\Delta$-domain will be called $\Delta$-analytic.
\end{definition}

\cfigure[8cm]{images/delta_domain.pdf}{A $\Delta$-domain}

\begin{thm}{Transfer theorem for $O$ and $o$}
	Let $\alpha, \beta \in \mathbb{R}$ and $f$ be a $\Delta$-analytic function.
	
	\begin{itemize}
		\item If $f$ is such that, in the some neighbourhood of 1 and its $\Delta$-domain, one has
		\[
		f(z) = O \left( {(1-z)}^{-\alpha} {\left( \log \frac{1}{1 - z} \right)}^\beta \right)
		\]
		Then
		\[
		[z^n] f(z) = O\left(n^{\alpha - 1} (\log n)^\beta\right)
		\]
		
		\item The same result holds, with $O$ replaced by $o$.
	\end{itemize}
\end{thm}

\end{document}