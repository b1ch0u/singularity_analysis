\documentclass[../main.tex]{subfiles}

\begin{document}

\chapter{Mathematical background}

\section{Notations}

\begin{definition}{}
	Let $f$ be a differentiable function. We note $f^{(k)}$ its $k$-th derivative.
\end{definition}

%\begin{definition}{{Open balls, closed balls, circles}}
%	For $z_0 \in \mathbb{C}$ and $r > 0$ we define
%	\begin{description}
%		\item[$\mathcal{B} (z_0, r)$] the open ball of center $z_0$ and radius~$r$ as
%		\[
%		\mathcal{B} (z_0, r) := \{z \in \mathbb{C} : |z - z_0| < r \}
%		\]
%		
%		\item[$\mathcal{B}_f (z_0, r)$] the closed ball of center $z_0$ and radius~$r$ as
%		\[
%		\mathcal{B}_f (z_0, r) := \{z \in \mathbb{C} : |z - z_0| \leq r \}
%		\]
%		
%		\item[$\mathcal{C} (z_0, r)$] the circle of center $z_0$ and radius~$r$ as
%		\[
%		\mathcal{C} (z_0, r) := \{z \in \mathbb{C} : |z - z_0| = r \}
%		\]
%	\end{description}
%\end{definition}

\section{Reminders from complex analysis}

%In all this section, $\Omega$ is an open subset of $\mathbb{C}$.

\begin{thm}{Leibniz rule}\label{Leibniz_rule}
	\[
	{(fg)}^{(n)} = \psum{i=0}^{n} {n \choose i} f^{(n - i)} g^{(i)}
	\]
\end{thm}

\begin{definition}{Analytic functions}
	A function $f$ is said to be \emph{analytic} on $\Omega$ if, for all $z_0 \in \Omega$, it admits an expansion
	\[
	f(z) = \psum{n \geq 0} f_n {(z - z_0)}^n
	\]
	that converges on some neighbourhood of $z_0$.
\end{definition}

\begin{thm}{Cauchy's integral formula}\label{thm_cauchy_integral}
	Let $\Omega$ be an open subset of $\mathbb{C}$.
	
	Let $\omega \in \Omega$ and $\rho > 0$ such that
	$\mathcal{B}_f (\omega, \rho) \subset \Omega$.
	
	Let $f$ be holomorphic on $\Omega$.
	
	Then for all $z_0 \in \mathcal{B} (\omega, \rho)$, we have
	\begin{equation*}
		f(z_0) = \frac{1}{2 i \pi} \int_{\mathcal{S} (\omega, r)} \frac{f(z)}{z - z_0} dz
	\end{equation*}
\end{thm}

\begin{cor}{Cauchy's coefficient formula}\label{thm_cauchy_coefficients}
	Let $f$ be analytic on some neighbourhood $\Omega$ of $z_0 \in \mathbb{C}$, and $r > 0$ such that $\mathcal{B}_f (z_0, r) \subset \Omega$, then for all $n$, one has
	\[
	f_n = \frac{1}{2i\pi} \int_{\mathcal{C} (z_0, r)}
	\frac{f(z)}{{(z - z_0)}^{n+1}} dz
	\]
\end{cor}

\begin{thm}{}
	Let $f$ be analytic in some neighbourhood $\Omega$ of $z_0 \in \mathbb{C}$, and $f_n$ such that on $\Omega$ one can write
	$$
	f(z) = \psum{n \geq 0} f_n {(z - z_0)}^n
	$$
	then for all $n$, we have
	$$
	f_n = \frac{f^{(n)}(z_0)}{n!}
	$$
\end{thm}


TODO

\subsection{Complex logarithm}

\paragraph{A few identities}

For a reference on the following definitions and identities, the reader is referred to \cite{Brown1996}.

From now on, except when explicitly stated otherwise, $\log z$ and $\arg z$ will stand for the principal determination of the complex logarithm and argument.

\begin{definition}{{$N_+, N_-$}}
	Let $z_1, z_2 \in \mathbb{C}^*$. Define $N_+(z_1, z_2)$ and $N_-(z_1, z_2)$ with
	\begin{align*}
	N_\pm =
	\begin{cases}
	-1 &\text{ if } \pi < \arg (z_1) \pm \arg (z_2)\\
	0  &\text{ if } - \pi < \arg (z_1) \pm \arg (z_2) \leq \pi \\
	1  &\text{ if } \arg (z_1) \pm \arg (z_2) \leq -\pi
	\end{cases}
	\end{align*}
\end{definition}

\begin{remark}
	The previous definition is intended to have to following relations hold:
	\begin{align*}
	\begin{cases}
	\arg(z_1 z_2) &= \arg (z_1) + \arg (z_2) + 2\pi N_+\\
	\arg \left(\frac{z_1}{z_2}\right) &= \arg (z_1) - \arg (z_2) + 2\pi N_-
	\end{cases}
	\end{align*}
\end{remark}

\begin{prop}{}
	Let $a, b, c \in \mathbb{C}^*$. Then
	\begin{align*}
		\log (a b) &= \log a + \log b + 2 i \pi N_+ (a, b)\\
		\log \left(\frac{a}{b}\right) &=  \log a - \log b + 2 i \pi N_- (a, b)\\
		{(a b)}^c &= a^c \times b^c \times e^{2i\pi c N_+(a, b)}\\
		{\left(\frac{a}{b}\right)}^c &= \frac{a^c}{b^c} e^{2i\pi c N_-(a, b)}
	\end{align*}
\end{prop}

\begin{cor}{}\label{cor_log_identities}
	Let $x \in \mathbb{R}^{+*}$ and $z,t \in \mathbb{C}^*$. Then $\arg(x) = 0$, so all classical identities over the real numbers extend identically:
	\begin{align*}
	\log (xz) &= \log x + \log z\\
	\log \left(\frac{x}{z}\right) &=  \log x - \log z\\
	{(x z)}^t &= x^t \times z^t\\
	{\left(\frac{x}{z}\right)}^t &= \frac{x^t}{z^t}
	\end{align*}
\end{cor}

%\section{D-finiteness and P-recursiveness}
%
%\begin{definition}{P-recursiveness}
%	Let ${(u_n)}_n$ a sequence of complex numbers.
%	
%	We say ${(a_n)}$ is \emph{P-recursive} if there exist $p_0, \dots, p_r \in \mathbb{C}[X]$ such that for all $n$,
%	$$
%	p_r(n) a_{n + r} + \dots + p_0(n) a_n = 0
%	$$
%\end{definition}
%
%\begin{thm}{A series is D-finite iif its coefficients sequence is P-recursive}
%	Let $g = \sum c_n z^n$ a formal sum. Then $g$ is D-finite if and only if ${(c_n)}$ is P-recursive.
%	
%	\tcblower
%	
%	TODO
%\end{thm}
%
%% TODO : Parler d'algèbres de Ore ?

\subfile{sections/basics_eq_diff}

\section{Location of Singularities}

%Let $g = \sum g_n z^n$ a formal sum.
%
%Then $g$ is solution of equation \ref{basic_eq_diff} if and only if its coefficients satisfy
%\begin{equation}
%p_r(z) \psum{n} g_{n + r} \frac{(n + r)!}{n!} z^n + \dots + p_0(z) \psum{n} g_n z^n = 0
%\end{equation}
%
%which is again equivalent to
%
%\begin{equation}
%\psum{n} \left[ p_r(z) g_{n + r} \frac{(n + r)!}{n!} + \dots + p_0(z) g_n \right] z^n = 0
%\end{equation}
%
%Let $c_{i, j}$ the coefficient of $p_i$ in $(X^j)$ and $d_i$ the degree of $p_i$.
%
%Then again, $g$ is solution of \ref{basic_eq_diff} if and only if
%\begin{equation}
%\left( \psum{j} c_{r, j} z^j \right) \psum{n} g_{n + r} \frac{(n + r)!}{n!} z^n
%+ \dots
%+ \left( \psum{j} c_{0, j} z^j \right) \psum{n} g_n z^n
%= 0
%\end{equation}
%
%Thus, $g$
%
%\begin{equation}
%\psum{n} \left[ \psum{j + k = n} c_{r, j} g_{k + r} \frac{(k + r)!}{k!} \right] z^n
%+ \dots
%+ \psum{n} \left[ \psum{j + k = n} c_{0, j} g_{k} \right] z^n
%= 0
%\end{equation}


\begin{thm}{Existence of a local basis of solutions}
	% TODO ref Wasow inaccessible
	
	Let $p_0, \dots, p_r \in \mathbb{C}[X]$ and $z_0 \in \mathbb{C}$ such that $p_r (z_0) \neq 0$.
	
	Then, in some neighbourhood of $z_0$, the equation
	\begin{equation}
	y^{(r)} + \frac{p_{r-1}}{p_r} y^{(r - 1)} + \dots + \frac{p_0}{p_r} y = 0
	\end{equation}
	admits a basis of $r$ analytic solutions.
	
	\tcblower
	
	The polynomial $p_r$ has finite degree, therefore has a finite number of roots. Since $p_r(z_0) \neq 0$, there is some neighbourhood $\Omega$ of $z_0$ where $p_r$ does not vanish.
	
	It follows that all $\frac{p_i}{p_r}$ are analytic on $\Omega$.
	
	Cauchy's theorem then applies and concludes the proof.
\end{thm}

\begin{cor}{Possible locations of singularities}\label{cor_sing_location}
	The only points where $f$ \emph{may} admit singularities are the zeros of $p_r$.
\end{cor}


\section{Structure theorems}

For further treatment of this section, one is referred to \cite{Wasow1965} (chapter II in particular).

\subsection{Another transformation}

Let $\zeta \in \mathbb{C}$, $y$ such that

\begin{equation*}
	y^{(r)} + a_{r - 1}(z) y^{(r - 1)} + \dots + a_0(z) y = 0
\end{equation*}

and define $Y : z \mapsto \begin{pmatrix}
y(z)\\
\vdots \\
{(z - \zeta)}^{r - 1} y^{(r - 1)}(z)
\end{pmatrix}$ (that is, $Y_i : z \mapsto {(z - \zeta)}^{i - 1} y^{(i - 1)}(z)$).

For all $i \leq r - 1$, we have $${(z - \zeta)} Y_i' = (i-1)Y_i + Y_{i + 1}$$ and
$$ {(z - \zeta)} Y	_r' = (r - 1)Y_r - {(z - \zeta)} a_{r-1} Y_{r - 1} - \dots - {(z - \zeta)}^r a_0 Y_1 $$

Then, equation \eqref{basic_eq_diff} is equivalent to

\begin{equation}\label{eq_diff_matrix_form}
{(z - \zeta)}Y' = A_\zeta(z)Y
\end{equation}

where $A_\zeta(z)$ is an $r \times r$ matrix.

\subsection{Regular singular points and indicial polynomials}


\begin{definition}{Regular singular points}
	Consider a differential equation (E) of the form \eqref{basic_eq_diff}.
	
	We say that $\zeta$ is a \emph{regular singular} point of (E), and $\zeta$ is a pole of $\frac{p_i}{p_r}$ of order at most $r - i$, for all $i \in [0, r - 1]$.
	
	Equivalently, $\zeta$ is a regular singular point if $A_\zeta(z)$ is analytic in some neighbourhood of $\zeta$, when one writes $(z - \zeta)Y' = A_\zeta(z) Y$.
\end{definition}

\begin{definition}{{Indicial polynomial, $I_\zeta$}}
	The characteristic polynomial of $A_\zeta(\zeta)$ is named the \emph{indicial polynomial} of equation \eqref{basic_eq_diff} and \eqref{eq_diff_matrix_form} at $\zeta$, denoted $I_\zeta$.
\end{definition}

%Before defining indicial polynomials in all generality, let us begin with a simple example, allowing one to get a grasp at where the definition comes from.
%
%\begin{exmpl}
%	Suppose we want to study
%	\begin{equation}\label{eqn_rd_walks_quarter_plane}
%	y^{(3)}
%	+ \frac{128 z^2 + 8 z - 6}{z (16 z^2 - 1)} y''
%	+ \frac{224 z^2 + 28 z - 6}{z^2 (16 z^2 - 1)} y'
%	+ \frac{64 z + 12}{z^2 (16 z^2 - 1)} y = 0
%	\end{equation}
%	
%	(For the interested reader, this equation is satisfied by the series counting the number of random walks in $\mathbb{N}^2$ with steps $(-1, 0), (1, 0), (0, -1), (0, 1)$).\\
%	
%	By corollary \ref{cor_sing_location}, a solution of equation \eqref{eqn_rd_walks_quarter_plane} may only admit singularities at $z = 0$ and $z = \pm \frac{1}{4}$.
%	One can see that 0 and $\pm \frac{1}{4}$ are regular singular points.
%	
%	Let us look for a solution of the form $y : z \mapsto z^\theta + o(z^\theta)$ around 0.
%	
%	Since $y^{(i)} : z \mapsto \theta (\theta - 1) \dots (\theta - i + 1) z^{\theta - i}$, one gets the equation
%	
%	\begin{align*}
%	z^2 (16 z^2 - 1) \theta (\theta - 1) (\theta - 2) z^{\theta - 3}\\
%	+ z(128 z^2 + 8 z - 6) \theta (\theta - 1) z^{\theta - 2}\\
%	+ (224 z^2 + 28 z - 6) \theta z^{\theta - 1}\\
%	+ (64 z + 12) z^\theta = 0
%	\end{align*}
%	
%	which can be rearranged
%	
%	\begin{align*}
%	\theta (\theta - 1) (\theta - 2) z^{\theta - 1}\\
%	+ (128 z^2 + 8 z - 6) \theta (\theta - 1) z^{\theta - 1}\\
%	+ (224 z^2 + 28 z - 6) \theta z^{\theta - 1}\\
%	+ (64 z + 12) z^\theta = 0
%	\end{align*}
%\end{exmpl}

\subsection{Results}


\begin{thm}{General structure theorem}
	Let $\zeta$ be a regular singular point of \eqref{basic_eq_diff}. No assumption is made on the roots of $I_\zeta$.
	
	Then, in a slit neighbourhood of $\zeta$, there exists a basis of solutions of the form
	\begin{equation}\label{general_structure_form}
	{(z - \zeta)}^{\theta_j} {(\log (z - \zeta))}^m H_j (z - \zeta)
	\end{equation}
	where $\theta_j$ are the roots of the indicial polynomial, each $H_j$ is analytic at 0, and $m \in \mathbb{N}$. 
\end{thm}

\subsection{Special cases}

\begin{definition}{G-functions}
	A formal series $f = \sum f_n z^n \in \mathbb{Q}[[z]]$ is called a \emph{G-function} if it is D-finite and there exists $C > 0$ such that for all $n$, we have
	\begin{align*}
	\begin{cases}
	|f_n| < C^n\\
	lcd(f_1, \dots, f_n) < C^n
	\end{cases}
	\end{align*}
	where $lcd(f_1, \dots, f_n)$ is the least common denominator of $f_1, \dots, f_n$.
\end{definition}

\begin{thm}{André-Chudnovsky-Katz Theorem}
	Let $f$ be a G-function. Then a minimal order annihilating D-finite equation for $f$ has only ordinary or regular singular points, and its indicial polynomial has only rational roots.
\end{thm}


\subfile{sections/transfer_theorems}
\subfile{sections/transfer_log_case}

\end{document}