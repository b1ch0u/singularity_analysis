\documentclass[../main.tex]{subfiles}

\begin{document}

\chapter{Mathematical background}

\section{Reminders from complex analysis}

TODO

%\section{D-finiteness and P-recursiveness}
%
%\begin{definition}{P-recursiveness}
%	Let ${(u_n)}_n$ a sequence of complex numbers.
%	
%	We say ${(a_n)}$ is \emph{P-recursive} if there exist $p_0, \dots, p_r \in \mathbb{C}[X]$ such that for all $n$,
%	$$
%	p_r(n) a_{n + r} + \dots + p_0(n) a_n = 0
%	$$
%\end{definition}
%
%\begin{thm}{A series is D-finite iif its coefficients sequence is P-recursive}
%	Let $g = \sum c_n z^n$ a formal sum. Then $g$ is D-finite if and only if ${(c_n)}$ is P-recursive.
%	
%	\tcblower
%	
%	TODO
%\end{thm}
%
%% TODO : Parler d'algèbres de Ore ?

\section{Singularities location}

%Let $g = \sum g_n z^n$ a formal sum.
%
%Then $g$ is solution of equation \ref{basic_eq_diff} if and only if its coefficients satisfy
%\begin{equation}
%p_r(z) \psum{n} g_{n + r} \frac{(n + r)!}{n!} z^n + \dots + p_0(z) \psum{n} g_n z^n = 0
%\end{equation}
%
%which is again equivalent to
%
%\begin{equation}
%\psum{n} \left[ p_r(z) g_{n + r} \frac{(n + r)!}{n!} + \dots + p_0(z) g_n \right] z^n = 0
%\end{equation}
%
%Let $c_{i, j}$ the coefficient of $p_i$ in $(X^j)$ and $d_i$ the degree of $p_i$.
%
%Then again, $g$ is solution of \ref{basic_eq_diff} if and only if
%\begin{equation}
%\left( \psum{j} c_{r, j} z^j \right) \psum{n} g_{n + r} \frac{(n + r)!}{n!} z^n
%+ \dots
%+ \left( \psum{j} c_{0, j} z^j \right) \psum{n} g_n z^n
%= 0
%\end{equation}
%
%Thus, $g$
%
%\begin{equation}
%\psum{n} \left[ \psum{j + k = n} c_{r, j} g_{k + r} \frac{(k + r)!}{k!} \right] z^n
%+ \dots
%+ \psum{n} \left[ \psum{j + k = n} c_{0, j} g_{k} \right] z^n
%= 0
%\end{equation}


\begin{thm}{Existence of a local basis of solutions}
	% TODO ref Wasow inaccessible
	
	Let $z_0 \in \mathbb{C}$ such that $p_r (z_0) \neq 0$.
	
	Then, in some neighbourhood of $z_0$, equation \ref{basic_eq_diff}
	admits a basis of $r$ analytic solutions.
	
	\tcblower
	
	We will show that a formal solution a uniquely defined by initial values of $f_i$. Then, that the thereby defined formal sum is analytic in some neighbourhood of $z_0$.
	
	\begin{center}
		***
	\end{center}

	Let $g = \sum g_n z^n \in \mathbb{C}[[X]]$.
	
	Then $g$ is solution of equation \ref{basic_eq_diff} if and only if its coefficients satisfy
	\begin{equation*}
	p_r(z) \psum{n} \frac{(n + r)!}{n!} g_{n + r} z^n + \dots + p_0(z) \psum{n} g_n z^n = 0
	\end{equation*}
	
	which is again equivalent to
	
	\begin{equation*}
	\psum{n} \left[ p_r(z) \frac{(n + r)!}{n!} g_{n + r} + \dots + p_0(z) g_n \right] z^n = 0
	\end{equation*}
	
	Now, by formal sum equality, we get a necessary and sufficient condition :
	
	\begin{equation}\label{rec_relation}
		\forall n, p_r(z_0) \frac{(n + r)!}{n!} g_{n + r} + \dots + p_0(z_0) g_n = 0
	\end{equation}
	
	Assuming $p_r(z_0) \neq 0$, a formal solution is therefore determined by its $r$ first coefficients, and we get a basis of $r$ formal solutions.
	
	\begin{center}
		***
	\end{center}
	
	Let $C_0 := \max\limits_{k \in [0, r - 1]} |g_k|$ and $M := \max \left| \frac{p_k(z_0)}{p_r(z_0)} \right|$.

	By equation \ref{rec_relation} and triangle inequality, one gets
	\begin{equation*}
		|g_r| \leq MrC_0
	\end{equation*}
	An immediate recurrence then yields $|g_{n + r}| \leq C_0 {(Mr)}^n$ for all $n$.
	
	So any formal solution's coefficients have at most exponential growth ; it follows that any formal solution is actually analytic in some neighbourhood of $z_0$.
\end{thm}

\begin{cor}{Possible locations of singularities}\label{cor_sing_location}
	The only points where $f$ \emph{may} admit singularities are the zeros of $p_r$.
\end{cor}


\section{Structure theorems}

For further treatment of this section, one is referred to \cite{Wasow1965} (chapter II in particular).

\subsection{Transformation into a matrix system}

Recall equation \ref{basic_eq_diff} :

\begin{equation*}
	u^{(r)} + \frac{p_{r-1}}{p_r} u^{(r - 1)} + \dots + \frac{p_0}{p_r} u = 0
\end{equation*}

Let us rename $a_i = \frac{p_i}{p_r}$.
Now define $Y : z \mapsto \begin{pmatrix}
u\\
x u'\\
\vdots \\
x^{r - 1} u^{(n - 1)}
\end{pmatrix}$ (that is, $y_i = x^{i - 1} u^{(i - 1)}$).\\

It is immediate that for all $i \leq r - 1$, we have $$z y_i' = (j-1)y_j + y_{j + 1}$$ and
$$ z y_r' = (r - 1)y_r - z a_{r-1} y_{r - 1} - \dots - z^r a_0 y_1 $$

Then, equation \ref{basic_eq_diff} is easily seen to be equivalent to

\begin{equation}\label{eq_diff_matrix_form}
zY' = A(z)Y
\end{equation}

where $A(z)$ is an $r \times r$ matrix.

\subsection{Regular singular points and indicial polynomials}


\begin{definition}{Regular singular points and Fuchsian condition}
	Let $f : z \mapsto \sum a_n z^n$ solution to \ref{basic_eq_diff}.
	
	We say $\zeta$ is a \emph{regular singular} point of $f$ if $\zeta$ is a pole of $\frac{p_i}{p_r}$ of order at most $r - i$, for all $i \in [0, r - 1]$.\\
	
	Equivalently, $\zeta$ is a regular singular point if $A(z)$ is analytic in some neighbourhood of $\zeta$, when one writes $(z - \zeta)Y' = A(z) Y$.\\
	
	A linear differential equation with only regular singular points is called \emph{Fuchsian}.
\end{definition}

\begin{definition}{{Indicial polynomial, $I_\zeta$}}
	The characteristic polynomial of $A(\zeta)$ is named the \emph{indicial polynomial} of equation \ref{basic_eq_diff} and \ref{eq_diff_matrix_form} at $\zeta$, written $I_\zeta$.
\end{definition}

%Before defining indicial polynomials in all generality, let us begin with a simple example, allowing one to get a grasp at where the definition comes from.
%
%\begin{exmpl}
%	Suppose we want to study
%	\begin{equation}\label{eqn_rd_walks_quarter_plane}
%	y^{(3)}
%	+ \frac{128 z^2 + 8 z - 6}{z (16 z^2 - 1)} y''
%	+ \frac{224 z^2 + 28 z - 6}{z^2 (16 z^2 - 1)} y'
%	+ \frac{64 z + 12}{z^2 (16 z^2 - 1)} y = 0
%	\end{equation}
%	
%	(For the interested reader, this equation is satisfied by the series counting the number of random walks in $\mathbb{N}^2$ with steps $(-1, 0), (1, 0), (0, -1), (0, 1)$).\\
%	
%	By corollary \ref{cor_sing_location}, a solution of equation \ref{eqn_rd_walks_quarter_plane} may only admit singularities at $z = 0$ and $z = \pm \frac{1}{4}$.
%	One can see that 0 and $\pm \frac{1}{4}$ are regular singular points. \\
%	
%	Let us look for a solution of the form $y : z \mapsto z^\theta + o(z^\theta)$ around 0.
%	
%	Since $y^{(i)} : z \mapsto \theta (\theta - 1) \dots (\theta - i + 1) z^{\theta - i}$, one gets the equation
%	
%	\begin{align*}
%	z^2 (16 z^2 - 1) \theta (\theta - 1) (\theta - 2) z^{\theta - 3}\\
%	+ z(128 z^2 + 8 z - 6) \theta (\theta - 1) z^{\theta - 2}\\
%	+ (224 z^2 + 28 z - 6) \theta z^{\theta - 1}\\
%	+ (64 z + 12) z^\theta = 0
%	\end{align*}
%	
%	which can be rearranged
%	
%	\begin{align*}
%	\theta (\theta - 1) (\theta - 2) z^{\theta - 1}\\
%	+ (128 z^2 + 8 z - 6) \theta (\theta - 1) z^{\theta - 1}\\
%	+ (224 z^2 + 28 z - 6) \theta z^{\theta - 1}\\
%	+ (64 z + 12) z^\theta = 0
%	\end{align*}
%\end{exmpl}

\subsection{Results}

\begin{thm}{Structure theorem without congruent roots}
	Let $\zeta$ be a singular regular point of \ref{basic_eq_diff}. Suppose $I_\zeta$ is such that no two roots differ by an integer (in particular, all roots are distinct).
	
	Then, in a slit neighbourhood of $\zeta$, there exists a basis of solutions with functions of the form
	\begin{equation}
	{(z - \zeta)}^{\theta_j} H_j (z - \zeta)
	\end{equation}
	where $\theta_j$ are the roots of the indicial polynomial and each $H_j$ is analytic at 0.
\end{thm}


\begin{thm}{General structure theorem}
	Let $\zeta$ be a singular regular point of \ref{basic_eq_diff}. No assumption is made on the roots of $I_\zeta$.
	
	Then, in a slit neighbourhood of $\zeta$, there exists a basis of solutions with functions of the form
	\begin{equation}\label{general_structure_form}
	{(z - \zeta)}^{\theta_j} \log^m (z - \zeta) H_j (z - \zeta)
	\end{equation}
	where $\theta_j$ are the roots of the indicial polynomial, each $H_j$ is analytic at 0, and $m$ is an integer. 
\end{thm}

\section{Transfer theorems}

\begin{thm}{Main transfer theorem}
	Let $f = \sum f_n z^n$ analytic in some open disk around the origin except for a single dominant singularity $\zeta$ and points in the ray	$R_\zeta = \{t \zeta : t \geq 1 \}$.\\
	
	If $f$ has a convergent expansion of the form \ref{general_structure_form} in a disk centred at $\zeta$ minus $R_\zeta$ and $f(z) \sim C {\left( 1 - \frac{z}{\zeta} \right)}^\alpha \log^m \left( 1 - \frac{z}{\zeta} \right)$ with $\alpha \not\in \mathbb{N}$,	
	then
	
	$$f_n \sim C {\zeta}^{-n} \frac{n^{-\alpha - 1}}{\Gamma(-\alpha)} \log^m (n)$$
\end{thm}

\end{document}