\documentclass[../main.tex]{subfiles}

\begin{document}

\chapter{Implementation}

Section \ref{section_implem_structure} presents the global structure of the algorithm, then details some parts. Section \ref{section_implem_tests} briefly presents the tests. Finally, section \ref{section_implem_perf} discusses performance aspects.

\section{Structure}\label{section_implem_structure}

We first present here the main algorithm, and the important sub-algorithms. The implementation is mostly transparent, except for a few optimizations like computing the local solutions in 0 only once.

\begin{algorithm}[H]
	\caption{Main algorithm}
	\SetAlgoLined
	\KwIn{$f$ defined by a $a_n \frac{d^n}{dz^n}f + \dots a_0 = 0$, initial coefficients~$f_0, \dots, f_n$, order and precision}
	\KwOut{Asymptotic expansion of $f$ with at least order terms, and coefficients with given precision}
	
	\Begin{
		
		Compute the roots of $a_n$ and group them by increasing module.
		
		Initialise the sum $S$ of contributions to the asymptotic expansion.
		
		\While{No contribution confirmed}{
			Load next group G of roots
			
			\ForEach{root $\rho \in$ G}{
				Compute the contribution of $\rho$.	
			}
			
			Sum contributions of G and add to $S$
		}
		
		\KwRet{S}
	}
	
\end{algorithm}

To compute the contribution of a specific root, we use the following algorithm

\begin{algorithm}[H]
	\caption{Computing the contribution of a root}
	\SetAlgoLined
	\KwIn{All entries of the main algorithm, and a specific root $\rho$}
	\KwOut{Contribution of $\rho$ to the asymptotic expansion of $f$}
	
	\Begin{
		Compute a basis $\mathcal{B}_0$ of solutions in 0, and decompose $f$ in $\mathcal{B}_0$.
		
		Compute a basis $\mathcal{B}_\rho$ of solutions in $\rho$, and decompose $f$ in $\mathcal{B}_\rho$.
		
		Initialize a sum S of terms contributions.
		
		\ForEach{term T in the local expansions}{
			Compute an asymptotic expansion of the Taylor coefficients of T.
			
			Add to S.
		}
	}
	\KwRet{S}
\end{algorithm}




\section{Tests}\label{section_implem_tests}

We give here a list of functions, along with an associated differential operator and an asymptotic expansion. We provide a name when the Taylor series is of combinatorial interest, and a reference for the not obvious/classical cases.

%\begin{landscape}
\begin{tiny}
\begin{center}
\begin{TAB}(r)[3pt]{|c|c|c|c|}{|c|c|c|c|c|c|c|c|c|c|c|c|c|c|}
	$f$ & Differential operator & Asymptotic expansion & Sequence name\\
	
	$\log(1 - z)$ & $(z - 1) Dz^2 + Dz$ & $\frac{-1}{n}$ & \\
	
	$\log(1 + z)$ & $(z + 1) Dz^2 + Dz$ & $\frac{(-1)^n}{n}$ & \\
	
	$\frac{1}{1-z}$ & $(z - 1) Dz + 1$ & 1 & \\
	
	$\frac{1}{1+z}$ & $(z - 1) Dz + 1$ & $(-1)^n$ & \\
	
	$\frac{z}{1-2z}$ & $z (1 - 2z) Dz - 1$ & $2^{n - 1}$ & \\
	
	$\frac{1}{1 - z^2}$ & $(1 - z^2) Dz - 2z$ & $\begin{cases}
			1 \text{ if $n$ is even}\\
			0 \text{ otherwise}
		\end{cases}$ & \\
	
	${(1-z)}^{3/2}$
	& $2 (z - 1) Dz - 3$
	& $\frac{1}{\sqrt{\pi n^5}} \left( \frac{3}{4} + \frac{45}{32n} + \frac{1155}{512n^2} + \dots \right)$
	& \cite{Flajolet2009}\\
	
	$\arctan(z)$ & $(1 + z^2)Dz^2 + 2 z Dz$ & $\begin{cases}
	\frac{(-1)^n}{2n+1} \text{ if $n$ is odd}\\
	0 \text{ otherwise}
	\end{cases}$ & \\

	$\frac{1}{1-z} \log\left(\frac{1}{1-z}\right)$
	& $(1 - z)^2 Dz^2 - 3 (1 - z) Dz + 1$
	& $\log(n) + \gamma + \frac{1}{2n} +o\left(\frac{1}{n}\right)$
	& Harmonic numbers\\

	$\frac{z}{1 - z - z^2}$ & $(1 - z - z^2) Dz^2 - (2 + 4z)Dz - 2$ & $\frac{\varphi^n - (-\varphi)^{-n}}{\sqrt{5}}$ & Fibonacci numbers\\
	
	$\frac{1 - \sqrt{1 - 4z}}{2z}$ & $(4z^2 - z)Dz^2+(10z-2)Dz+2$ & $\frac{4^n}{\sqrt{\pi n}} \left(1 - \frac{1}{8n} + \frac{1}{128n^2} + \dots \right)$ & Catalan numbers\\
	
	${\scriptscriptstyle\frac{1 - z - \sqrt{1-2z-3z^2}}{2z}}$ & 
	${\scriptscriptstyle (3z^4 + 2z^3 - z^2)Dz^2 + (6z^3 + 3z^2 - z)Dz + 1 }$ & ${\scriptstyle \frac{\sqrt{3}}{2\sqrt{\pi}n^{3/2}} 3^n \left(1 - \frac{15}{16n} + \frac{505}{512n^2} + \dots \right)}$ &
	Motzkin numbers \cite{Flajolet2009}\\
	
	& $(4z^2 - z)Dz^2 + (14z - 2)Dz + 6$ & $\frac{4^{n}}{\pi n} \left( 4 - \frac{6}{n} + \frac{19 - 2(-1)^n}{2n^2} + \dots \right)$
		& Walks in $\mathbb{N}^2$ \cite{Melczer2020}\\
\end{TAB}
\end{center}
\end{tiny}

%A l'ordre 6 et 7:
%random walks in quadrant : ([1.27324 +/- 1.98e-6] + [+/- 1.57e-7]*I)*4^n*n^(-1) + ([-1.90986 +/- 2.96e-6] + [+/- 2.35e-7]*I)*4^n*n^(-2) + ([0.318310 +/- 7.23e-7])*4^n*n^(-3)*(e^(I*arg(-1)))^n + ([3.0239 +/- 5.09e-5] + [+/- 8.42e-7]*I)*4^n*n^(-3) + ([-1.4324 +/- 1.84e-5])*4^n*n^(-4)*(e^(I*arg(-1)))^n + O(4^n*n^(-4))
%git:(master) 
%seb@seb-HP-EliteBook-820-G2:pts/0->/home/seb/Dropbox/M2 fonda/mezzarobba/depot_git (0)
%·> sage tests.sage
%
%random walks in quadrant : ([1.27324 +/- 1.98e-6] + [+/- 1.57e-7]*I)*4^n*n^(-1) + ([-1.90986 +/- 2.96e-6] + [+/- 2.35e-7]*I)*4^n*n^(-2) + ([0.318310 +/- 7.23e-7])*4^n*n^(-3)*(e^(I*arg(-1)))^n + ([3.0239 +/- 5.09e-5] + [+/- 8.42e-7]*I)*4^n*n^(-3) + ([-1.4324 +/- 1.84e-5])*4^n*n^(-4)*(e^(I*arg(-1)))^n + ([-5.0134 +/- 7.11e-5] + [+/- 4.26e-6]*I)*4^n*n^(-4) + O(4^n*n^(-5)*log(n))
%git:(master) 


\section{Performance}\label{section_implem_perf}
TODO



\end{document}