\documentclass[../main.tex]{subfiles}

\begin{document}

\chapter{Introduction}


Pick your favourite combinatorial construction. Let $a_n$ the number of such structures of size $n$. We wish to be able to compute an asymptotic expansion of $a_n$ automatically.

Let $f = \sum a_n z^n$ be the complex series associated to $(a_n)$.

We shall see that, if $f$ has a positive convergence radius, then one has asymptotically $a_n = A^n \theta(n)$ where $\theta$ has sub-exponential growth.
Two principles, stated in \cite{Flajolet2009}, shall guide one's search :

\begin{itemize}
	\item	\emph{First Principle of Coefficient Asymptotics} : The location of a function’s
			singularities dictates the exponential growth ($A^n$) of its coefficients.
	
	\item	\emph{Second Principle of Coefficient Asymptotics} : The nature of a function’s
			singularities determines the associate subexponential factor ($\theta(n)$).
	
\end{itemize}
%Almost everything that we present here is, to some degree, borrowed from \cite{Flajolet2009}. Flajolet and Sedgewick amply describe these techniques, along with many examples.

\section{Mathematical sketch}

\paragraph{From a combinatorial problem to a differential equation}
Powerful techniques exist to translate a combinatorial construction into a \emph{D-finite} relation, namely a differential equation with polynomial coefficients.
We will not cover those techniques here. If interested, one is referred to \cite{Flajolet2009}.

From now on, we will assume that a non trivial D-finite relation satisfied by $f$ is given, which is 
\begin{equation}\label{basic_eq_diff}
y^{(r)} + \frac{p_{r-1}}{p_r} y^{(r - 1)} + \dots + \frac{p_0}{p_r} y = 0
\end{equation}
where $p_0, \dots, p_r \in \mathbb{C}[X]$.

\paragraph*{Singularities location}
We will first see that $f$ may only have singularities at roots of $p_r$. Thereafter, we define $\Xi := \{ \text{roots of } p_r\}$. If $f$ has at least one singularity, minimal ones (by module) are called \emph{dominant singularities}.

\paragraph*{Local basis structure theorems}
Following the definition of \emph{regular singular points}, where some technical condition is satisfied, we prove that, in a \emph{slit} neighbourhood of any such point $\zeta$, equation \eqref{basic_eq_diff} admits a local basis of solutions of the form $${(z - \zeta)}^{\theta_j} \log^m (z - \zeta) H_j (z - \zeta)$$ with $H_j$ analytic at 0. This basis can be explicitly computed. 

\paragraph*{Transfer theorems}
We then investigate \emph{transfer theorems}. Assume $f$ has at least one singularity, and  all dominant singularities are regular singular points. After expressing $f$ in the previous form around all dominant singularities, transfer theorems allow one to compute an asymptotic expansion of $f_n$.


\section{Implementation overview}

TODO

\end{document}