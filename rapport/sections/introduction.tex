\documentclass[../main.tex]{subfiles}

\begin{document}

\chapter{Introduction}


Pick your favourite combinatorial construction. Let $a_n$ the number of such structures of size $n$. We wish to be able to compute an asymptotic expansion of $a_n$ automatically.

Let $f = \sum a_n z^n$ be the complex series associated to $(a_n)$.

We shall see that, if $f$ has a positive convergence radius, then one has asymptotically $a_n = A^n \theta(n)$ where $\theta$ has sub-exponential growth.
Two principles, stated in \cite{Flajolet2009}, shall guide one's search :

\begin{itemize}
	\item	\emph{First Principle of Coefficient Asymptotics} : The location of a function’s
			singularities dictates the exponential growth ($A^n$) of its coefficients.
	
	\item	\emph{Second Principle of Coefficient Asymptotics} : The nature of a function’s
			singularities determines the associate subexponential factor ($\theta(n)$).
	
\end{itemize}
%Almost everything that we present here is, to some degree, borrowed from \cite{Flajolet2009}. Flajolet and Sedgewick amply describe these techniques, along with many examples.

\section{State of the art}

%It has been known since TODO that 

In 1990, Flajolet and Odlyzko \cite{FlajoletOdlyzko1990} proved transfer theorems, that allow one to 
TODO %TODO


\section{Mathematical sketch}

\paragraph{From a combinatorial problem to a differential equation}
Powerful techniques exist to translate a combinatorial construction into a \emph{D-finite} relation, namely a differential equation with polynomial coefficients.
We will not cover those techniques here. If interested, one is referred to \cite{Flajolet2009}.

From now on, we will assume that a non trivial D-finite relation satisfied by $f$ is given, which is a relation of the form
\begin{equation}\label{basic_eq_diff}
y^{(r)} + \frac{p_{r-1}}{p_r} y^{(r - 1)} + \dots + \frac{p_0}{p_r} y = 0
\end{equation}
where $p_0, \dots, p_r \in \mathbb{C}[X]$.

\begin{definition}{D-finite (holonomic) function}
	A function satisfying a D-finite relation will be said D-finite itself, or \emph{holonomic}.
\end{definition}

\paragraph*{Singularities location}
We will first see that $f$ may only have singularities at roots of $p_r$. Thereafter, we define $\Xi := \{ \text{roots of } p_r\}$. If $f$ has at least one singularity, minimal ones (by module) are called \emph{dominant singularities}.

\paragraph*{Local basis structure theorems}
Following the definition of \emph{regular singular points}, where some technical condition is satisfied, it can be proved that, in a \emph{slit} neighbourhood of any such point $\zeta$, equation \eqref{basic_eq_diff} admits a local basis of solutions of the form $${(z - \zeta)}^{\theta_j} \log^m (z - \zeta) H_j (z - \zeta)$$ with $H_j$ analytic at 0. This basis can be explicitly computed. 

\paragraph*{Transfer theorems}
We then investigate \emph{transfer theorems}. Assume $f$ has at least one singularity, and  all dominant singularities are regular singular points. After expressing $f$ in the previous form around all dominant singularities, transfer theorems allow one to compute an asymptotic expansion of $f_n$.


\section{Implementation overview}

The implementation is in SageMath, mostly (and vastly) relying on the \verb|ore_algebra| and \verb|AsymptoticRing| modules.

\paragraph{\py{extract_asymptotics}}

A function \py{extract_asymptotics} is implemented, with the following definition:

\begin{pyblock}
def extract_asymptotics(op,
						first_coefficients,
						z,
						order=DEFAULT_ORDER,
						precision=DEFAULT_PRECISION) -> expr
\end{pyblock}

For a holonomic function $f$, \py{extract_asymptotics} takes

\begin{itemize}
	\item A differential operator \py{op}, such that $op \cdot f = 0$
	
	\item A list \py{first_coefficients} of the first Taylor coefficients of $f$
	
	\item The variable \py{z} used in definition of \py{op}, only for technical reasons
	
	\item The desired \py{order} of the asymptotic expansion
	
	\item The desired certified \py{precision} for the constants
\end{itemize}

It returns an asymptotic expansion of the coefficients of $f$, up to the desired \py{order} and with constants certified at least with the given \py{precision}.

\paragraph{Global structure}

We first locate the roots of $p_r$, and group them by increasing module.

Then, as long as no root has been \textit{proved} to be a singularity of $f$, we iterate through the groups and sum their contributions. A root of $p_r$ can indeed not always be proved to be a singularity of $f$ only by computing the coefficients of $f$ in the local basis: if one of these coefficients is precisely 0, successive approximations will never be able to distinguish it from 0, yet not proving either that it would be nil.

Then for each root, we make use of \py{local_basis_expansions}, defined in the \py{ore_algebra} module. That function allows one to compute a local basis of solutions to \py{op}, along with their expansion up to any desired order. A call to \py{numerical_transition_matrix} then allows us to determine the expression of $f$ in that local basis, with constants certified to the desired \py{precision}.

Calling \py{SingularityAnalysis} (a specially crafted function from the \py{asymptotic_ring} module) on each term appearing and summing the results finally yields the desired expansion.

\end{document}